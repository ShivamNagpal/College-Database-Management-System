\chapter{Implementation}
\setstretch{1}
\section{Code For Major Functionalities}
\subsection{Database Operation}
\thispagestyle{fancy}

\begin{lstlisting}[caption=DatabaseContract.java]
public class DatabaseContract {
  public static final String ROW_ID = "_ROWID";
  public static final String ROW_ID_DATA_TYPE = "BIGINT(8) UNSIGNED NOT NULL UNIQUE AUTO_INCREMENT";
  public static final String PROCEDURE_IA_MARKS_CALCULATE_AVERAGE = "IA_MARKS_CALCULATE_AVERAGE";
  public static final String SQL_IA_MARKS_CALCULATE_AVERAGE = "CREATE PROCEDURE " + PROCEDURE_IA_MARKS_CALCULATE_AVERAGE + "()\n" +
      "BEGIN\n" +
      " DECLARE a, b, c, average INT;\n" +
      " DECLARE done INT DEFAULT FALSE;\n" +
      " DECLARE id BIGINT(8);\n" +
      " DECLARE cur CURSOR FOR\n" +
      " SELECT " + DatabaseContract.ROW_ID + ", " + DatabaseContract.IaMarks.TEST1 + ", " + DatabaseContract.IaMarks.TEST2 + ", " + DatabaseContract.IaMarks.TEST3 + " FROM " + DatabaseContract.IaMarks.TABLE_NAME + ";\n" +
      " DECLARE CONTINUE HANDLER FOR NOT FOUND SET done = TRUE;  \n" +
      " OPEN cur;\n" +
      "   m_loop : LOOP\n" +
      "     SET done = FALSE;\n" +
      "     FETCH cur INTO id, a, b, c;\n" +
      "     IF(done) THEN\n" +
      "       LEAVE m_loop;\n" +
      "     END IF;\n" +
      "     SET average = (a+b+c)/3;\n" +
      "     UPDATE college.IA_MARKS SET AVG_MARKS = average WHERE " + DatabaseContract.ROW_ID + " = id;\n" +
      "     SET done = done+1;\n" +
      "   END LOOP;\n" +
      " CLOSE cur;\n" +
      "END\n";

  public class Professor {
    public static final String TABLE_NAME = "PROFESSOR";
    public static final String NAME = "NAME";
    public static final String PROFESSOR_ID = "PROFESSOR_ID";
    public static final String DATE_OF_BIRTH = "DATE_OF_BIRTH";
    public static final String ADDRESS = "ADDRESS";
    public static final String PHONE = "PHONE";
    public static final String EMAIL = "EMAIL";
    public static final String DEPARTMENT_ID = "DEPARTMENT_ID";
    public static final String DESIGNATION = "DESIGNATION";
  }

  public class Department {
    public static final String TABLE_NAME = "DEPARTMENT";
    public static final String NAME = "NAME";
    public static final String DEPARTMENT_ID = "DEPARTMENT_ID";
  }

  public class Student {
    public static final String TABLE_NAME = "STUDENT";
    public static final String NAME = "NAME";
    public static final String STUDENT_ID = "STUDENT_ID";
    public static final String DATE_OF_BIRTH = "DATE_OF_BIRTH";
    public static final String ADDRESS = "ADDRESS";
    public static final String PHONE = "PHONE";
    public static final String EMAIL = "EMAIL";
    public static final String DEPARTMENT_ID = "DEPARTMENT_ID";
  }

  public class Subject {
    public static final String TABLE_NAME = "SUBJECT";
    public static final String NAME = "NAME";
    public static final String SUBJECT_ID = "SUBJECT_ID";
    public static final String SCHEME = "SCHEME";
    public static final String SEMESTER = "SEMESTER";
    public static final String SEMESTER_IDX = "SEMESTER_IDX";
    public static final String CREDITS = "CREDITS";
    public static final String CREDITS_IDX = "CREDITS_IDX";
    public static final String DEPARTMENT_ID = "DEPARTMENT_ID";
 }

  public class SemesterSection {
    public static final String TABLE_NAME = "SEMESTER_SECTION";
    public static final String SEM_SEC_ID = "SEM_SEC_ID";
    public static final String SEMESTER = "SEMESTER";
    public static final String SEMESTER_IDX = "SEMESTER_IDX";
    public static final String SECTION = "SECTION";
  }

  public class Division {
    public static final String TABLE_NAME = "DIVISION";
    public static final String STUDENT_ID = "STUDENT_ID";
    public static final String SEM_SEC_ID = "SEM_SEC_ID";
  }

  public class Teaches {
    public static final String TABLE_NAME = "TEACHES";
    public static final String PROFESSOR_ID = "PROFESSOR_ID";
    public static final String SEM_SEC_ID = "SEM_SEC_ID";
    public static final String SUBJECT_ID = "SUBJECT_ID";
  }

  public class IaMarks {
    public static final String TABLE_NAME = "IA_MARKS";
    public static final String STUDENT_ID = "STUDENT_ID";
    public static final String SEM_SEC_ID = "SEM_SEC_ID";
    public static final String SUBJECT_ID = "SUBJECT_ID";
    public static final String TEST1 = "TEST1";
    public static final String TEST1_IDX = "TEST1_IDX";
    public static final String TEST2 = "TEST2";
    public static final String TEST2_IDX = "TEST2_IDX";
    public static final String TEST3 = "TEST3";
    public static final String TEST3_IDX = "TEST3_IDX";
    public static final String AVG_MARKS = "AVG_MARKS";
    public static final String AVG_MARKS_IDX = "AVG_MARKS_IDX";
  }
}
\end{lstlisting}

\begin{lstlisting}[caption=DatabaseHelper.java]
package com.nagpal.shivam.dbms.data;

import com.nagpal.shivam.dbms.Log;
import com.nagpal.shivam.dbms.data.DatabaseContract.*;
import com.nagpal.shivam.dbms.model.*;

import java.sql.*;
import java.util.ArrayList;
import java.util.List;

public class DatabaseHelper {


  public static final String SEARCH_MODE_NATURAL_LANGUAGE = "Natural Language Mode";
  public static final String SEARCH_MODE_BOOLEAN = "Boolean Mode";
  private static final String CLASS_NAME = DatabaseHelper.class.getSimpleName();

  public static List<DepartmentData> fetchDepartmentDetails() {
    String sql = "SELECT * FROM " +
        Department.TABLE_NAME;
    Connection connection = Database.getConnection();
    List<DepartmentData> list = new ArrayList<>();
    try {
      Statement statement = connection.createStatement();
      ResultSet set = statement.executeQuery(sql);
      int rowIdIndex = set.findColumn(DatabaseContract.ROW_ID);
      int nameIndex = set.findColumn(Department.NAME);
      int departmentIdIndex = set.findColumn(Department.DEPARTMENT_ID);

      while (set.next()) {
        list.add(new DepartmentData(set.getLong(rowIdIndex), set.getString(nameIndex), set.getString(departmentIdIndex)));
      }
    } catch (SQLException e) {
      Log.e(CLASS_NAME, e.getMessage());
    }
    return list;
  }

  public static List<DepartmentData> fetchParticularDepartment(String departmentId, boolean include) {
    String operator = include ? " = " : " != ";
    String sql = "SELECT * FROM " +
        Department.TABLE_NAME +
        " WHERE " +
        Department.DEPARTMENT_ID + operator + "'" + departmentId + "'";
    Connection connection = Database.getConnection();
    List<DepartmentData> list = new ArrayList<>();
    try {
      Statement statement = connection.createStatement();
      ResultSet set = statement.executeQuery(sql);
      int rowIdIndex = set.findColumn(DatabaseContract.ROW_ID);
      int nameIndex = set.findColumn(Department.NAME);
      int departmentIdIndex = set.findColumn(Department.DEPARTMENT_ID);

      while (set.next()) {
        list.add(new DepartmentData(set.getLong(rowIdIndex), set.getString(nameIndex), set.getString(departmentIdIndex)));
      }
    } catch (SQLException e) {
      Log.e(CLASS_NAME, e.getMessage());
    }
    return list;
  }

  public static List<DepartmentData> searchDepartmentDetails(String searchString, String mode) {
    String sql = "SELECT * FROM " +
        Department.TABLE_NAME +
        " WHERE MATCH(" + Department.NAME + "," + Department.DEPARTMENT_ID + ") " +
        "AGAINST(? IN " + mode.toUpperCase() + ")";
    Connection connection = Database.getConnection();
    List<DepartmentData> list = new ArrayList<>();
    try {
      PreparedStatement preparedStatement = connection.prepareStatement(sql);
      preparedStatement.setString(1, searchString);
      ResultSet set = preparedStatement.executeQuery();
      int rowIdIndex = set.findColumn(DatabaseContract.ROW_ID);
      int nameIndex = set.findColumn(Department.NAME);
      int departmentIdIndex = set.findColumn(Department.DEPARTMENT_ID);

      while (set.next()) {
        list.add(new DepartmentData(set.getLong(rowIdIndex), set.getString(nameIndex), set.getString(departmentIdIndex)));
      }
    } catch (SQLException e) {
      Log.e(CLASS_NAME, e.getMessage());
    }
    return list;
  }

  public static int updateDepartment(DepartmentData departmentData) {
    String sql = "UPDATE " +
        Department.TABLE_NAME +
        " SET " +
        Department.NAME + "=?, " +
        Department.DEPARTMENT_ID + "=? " +
        "WHERE " + DatabaseContract.ROW_ID + " =" + departmentData.rowId;
    Connection connection = Database.getConnection();
    try {
      PreparedStatement statement = connection.prepareStatement(sql);
      statement.setString(1, departmentData.name);
      statement.setString(2, departmentData.departmentId);
      statement.executeUpdate();
    } catch (SQLException e) {
      Log.e(CLASS_NAME, e.getMessage());
      return e.getErrorCode();
    }
    return SqlErrorCodes.SQLITE_OK;
  }

  public static int deleteRow(String tableName, long rowId) {
    String sql = "DELETE FROM " +
        tableName +
        " WHERE " + DatabaseContract.ROW_ID + " =" + rowId;
    Connection connection = Database.getConnection();
    try {
      Statement statement = connection.createStatement();
      statement.execute(sql);
    } catch (SQLException e) {
      Log.e(CLASS_NAME, e.getMessage());
      return e.getErrorCode();
    }
    return SqlErrorCodes.SQLITE_OK;
  }

  public static int executeIaMarksCalculateAverage() {
    String sql = "{CALL " + DatabaseContract.PROCEDURE_IA_MARKS_CALCULATE_AVERAGE + "()}";
    Connection connection = Database.getConnection();
    try {
      CallableStatement statement = connection.prepareCall(sql);
      statement.execute();
    } catch (SQLException e) {
      Log.e(CLASS_NAME, e.getMessage());
      return e.getErrorCode();
    }
    return SqlErrorCodes.SQLITE_OK;
  }
}

\end{lstlisting}


\pagebreak
\subsection{Model Class}
\thispagestyle{fancy}
\begin{lstlisting}[caption=DepartmentData.java]
package com.nagpal.shivam.dbms.model;

import com.nagpal.shivam.dbms.data.PreviewIgnoredAttribute;

public class DepartmentData {
  @PreviewIgnoredAttribute
  public long rowId;
  public String name;
  public String departmentId;

  public DepartmentData() {
  }

  public DepartmentData(String name, String departmentId) {
    this.name = name;
    this.departmentId = departmentId;
  }

  public DepartmentData(long rowId, String name, String departmentId) {
    this.rowId = rowId;
    this.name = name;
    this.departmentId = departmentId;
  }
  
}
\end{lstlisting}


\pagebreak
\subsection{UI Scenes}
\thispagestyle{fancy}
\begin{lstlisting}[caption=PreviewDatabase.java]
package com.nagpal.shivam.dbms.ui;

import com.nagpal.shivam.dbms.navigation.Intent;
import com.nagpal.shivam.dbms.navigation.NavUtil;
import javafx.scene.Scene;
import javafx.scene.control.Button;
import javafx.scene.image.Image;
import javafx.scene.image.ImageView;
import javafx.scene.layout.*;

import java.util.ArrayList;
import java.util.LinkedHashMap;
import java.util.Map;

import static com.nagpal.shivam.dbms.Main.sStage;

public class PreviewDatabase extends UiScene {
  @Override
  public void setScene() {
    Pane pane = getLayout();
    pane.getStyleClass().add("parentPane");
    Scene scene = new Scene(pane);
    scene.getStylesheets().add("css/PreviewDatabaseScene.css");
    sStage.setTitle("Database Preview");
    sStage.setScene(scene);
    pane.requestFocus();
  }

  @Override
  protected Pane getLayout() {
    Image logo = new Image("images/college-pic.png");
    ImageView logoImageView = new ImageView(logo);
    logoImageView.setFitWidth(400);
    logoImageView.setFitHeight(300);
    FlowPane logoImageViewFlowPane = new FlowPane(logoImageView);
    logoImageViewFlowPane.getStyleClass().add("logoImageViewFlowPane");

    LinkedHashMap<String, Class> linkedHashMap = new LinkedHashMap<>();
    linkedHashMap.put("Professor", PreviewProfessor.class);
    linkedHashMap.put("Student", PreviewStudent.class);
    linkedHashMap.put("Department", PreviewDepartment.class);
    linkedHashMap.put("Subject", PreviewSubject.class);
    linkedHashMap.put("Division", PreviewDivision.class);
    linkedHashMap.put("Teaches", PreviewTeaches.class);
    linkedHashMap.put("Semester-Section", PreviewSemesterSection.class);
    linkedHashMap.put("IaMarks", PreviewIaMarks.class);
    TilePane tilePane = new TilePane();

    ArrayList<Button> buttons = new ArrayList<>();
    for (Map.Entry<String, Class> entry : linkedHashMap.entrySet()) {
      Button button = new Button(entry.getKey());
      button.setOnAction(event -> NavUtil.startScene(new Intent(PreviewDatabase.this, entry.getValue())));
      buttons.add(button);
    }

    tilePane.setPrefColumns(3);
    tilePane.setHgap(20);
    tilePane.setVgap(20);
    tilePane.getChildren().addAll(buttons);

    FlowPane tileFlowPane = new FlowPane(tilePane);
    tileFlowPane.getStyleClass().add("tileFlowPane");

    VBox vBox = new VBox(logoImageViewFlowPane, tileFlowPane);
    vBox.setSpacing(30);

    BorderPane borderPane = new BorderPane();
    borderPane.setCenter(vBox);
    return borderPane;
  }
}

\end{lstlisting}

\pagebreak
\subsection{CSS}
\thispagestyle{fancy}
\begin{lstlisting}[caption=PreviewDatabase.css]
.parentPane {
    -fx-min-width: 800;
    -fx-min-height: 600;
}

.logoImageViewFlowPane {
    -fx-alignment: center;
}

.tileFlowPane {
    -fx-alignment: center;
}

.button {
    -fx-pref-width: 200;
    -fx-background-radius: 100;
    -fx-padding: 20;
    -fx-alignment: center;
}

\end{lstlisting}

