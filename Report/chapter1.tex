\chapter{Introduction}

\section{Database Technologies}
The essential feature of database technology is that it provides an internal representation (model) of the external world of interest. Examples are the representation of a particular date / time / flight / aircraft in airline reservation or of item code/item description/quantity on hand/reorder level / reorder quantity in a stock control system. The technology involved is concerned primarily with maintaining the internal representation consistent with external reality; this involves the results of extensive Research and Development over the past 30 years in areas such as user requirements analysis, data modeling, process modeling, data integrity, concurrency, transactions, file organization, indexing, rollback and recovery, persistent programming, object-orientation, logic programming, deductive database systems, active database systems... and in all these (and other) areas there remains much to be done. The essential point is that database technology is a CORE TECHNOLOGY with links to:
\begin{itemize}
\item{Information management / processing}
\item{Data analysis / statistics}
\item{Data visualization / presentation}
\item{Multimedia and hypermedia}
\item{Office and document systems}
\item{Business processes, workflow, CSCW (computer-supported cooperative work)}
\end{itemize}
Relational DBMS is the modern base technology for many business applications. It offers flexibility and easy-to-use tools at the expense of ultimate performance. More recently relational systems have started to extend their facilities in the directions of information retrieval, object-orientation and deductive/active systems leading to the so-called 'Extended
Relational Systems'.

Information Retrieval Systems started with handling library catalogues and extended to full free-text utilizing inverted index technology with a lexicon or thesaurus. Modern systems utilize some KBS (knowledge-based systems) techniques to improve retrieval.

Object-Oriented DBMS started for engineering applications where objects are complex,
have versions and need to be treated as a complete entity. OODBMSs share many of the OOPL features such as identity, inheritance, late binding, overloading and overriding.
OODBMSs have found favour in engineering and office systems but have not yet been successful in traditional application areas. Deductive / Active DBMS have emerged over the last 20 years and combine logic programming technology with database technology. This allows the database itself to react to external events to maintain dynamically its integrity with respect to the real world.

\thispagestyle{fancy}

\section{Characteristics of Database Approach }
Traditionally, data was organized in file formats. DBMS was a new concept then, and all the research was done to make it overcome the deficiencies in traditional style of data management. A modern DBMS has the following characteristics:
\begin{itemize}
\item{Real-world entity} - A modern DBMS is more realistic and uses real-world entities to design its architecture. It uses the behavior and attributes too. For example, a school database may use students as an entity and their age as an attribute.
\item{Relation-based tables} - DBMS allows entities and relations among them to form tables. A user can understand the architecture of a database just by looking at the table names.
\item{Isolation of data and application} - A database system is entirely different than its data. A database is an active entity, whereas data is said to be passive, on which the database works and organizes. DBMS also stores metadata, which is data about data, to ease its own process.
\item{Less redundancy} - DBMS follows the rules of normalization, which splits a relation when any of its attributes is having redundancy in values. Normalization is a mathematically rich and scientific process that reduces data redundancy.
\item{Consistency} - Consistency is a state where every relation in a database remains consistent. There exist methods and techniques, which can detect attempt of leaving database in inconsistent state. A DBMS can provide greater consistency as compared to earlier forms of data storing applications like file-processing systems.
\item{Query Language} - DBMS is equipped with query language, which makes it more efficient to retrieve and manipulate data. A user can apply as many and as different filtering options as required to retrieve a set of data. Traditionally it was not possible where file-processing system was used.
\item{ACID Properties} - DBMS follows the concepts of Atomicity, Consistency, Isolation, and Durability (normally shortened as ACID). These concepts are applied on transactions, which manipulate data in a database. ACID properties help the database
stay healthy in multi-transactional environments and in case of failure.
\item{Multiuser and Concurrent Access} - DBMS supports multi-user environment and allows them to access and manipulate data in parallel. Though there are restrictions on transactions when users attempt to handle the same data item, but users are always unaware of them.
\item{Multiple views} - DBMS offers multiple views for different users. A user who is in the Sales department will have a different view of database than a person working in the Production department. This feature enables the users to have a concentrate view of the database according to their requirements.
\item{Security} - Features like multiple views offer security to some extent where users are unable to access data of other users and departments. DBMS offers methods to impose constraints while entering data into the database and retrieving the same at a later stage. DBMS offers many different levels of security features, which enables multiple users to have different views with different features. For example, a user in the Sales department cannot see the data that belongs to the Purchase department.
Additionally, it can also be managed how much data of the Sales department should be displayed to the user. Since a DBMS is not saved on the disk as traditional file systems, it is very hard for miscreants to break the code.
\end{itemize}

\thispagestyle{fancy}

\section{Applications of DBMS}
Applications where we use Database Management Systems are: \\
\begin{itemize}
\item \textbf{Telecom:} There is a database to keeps track of the information regarding calls made, network usage, customer details etc. Without the database systems it is hard to
maintain that huge amount of data that keeps updating every millisecond.
\item \textbf{Industry:} Where it is a manufacturing unit, warehouse or distribution centre, each one needs a database to keep the records of ins and outs. For example distribution
centre should keep a track of the product units that supplied into the centre as well as
the products that got delivered out from the distribution centre on each day; this is
where DBMS comes into picture.
\item \textbf{Banking System: } For storing customer info, tracking day to day credit and debit
transactions, generating bank statements etc. All this work has been done with the help
of Database management systems.
\item \textbf{Education Sector: } Database systems are frequently used in schools and colleges to store and retrieve the data regarding student details, staff details, course details, exam
details, payroll data, attendance details, fees details etc. There is a hell lot amount of
inter-related data that needs to be stored and retrieved in an efficient manner.
\item \textbf{Online Shopping: }You must be aware of the online shopping websites such as
Amazon, Flip kart etc. These sites store the product information, your addresses and
preferences, credit details and provide you the relevant list of products based on your
query. All this involves a Database management system.
\end{itemize}
\thispagestyle{fancy}
\newpage
\section{Problem Description/Statement}
Amusement parks get thousands of customers of varying age groups every single day.
Each individual customer will have a unique taste and accordingly chooses an attraction in the park.
This wealth of data generally goes unstored and unused.
In this project we have tried to store the activities of each customer once he enters the park.
This data can be crucial to know about the areas of improvement, which will benefit the owners by generating more revenue.

We have automated the tasks of adding customers to the park, since it is tedious to enter the details of each customer manually,
not only is that task time-consuming, it is also impractical to do statistical analysis on minimal data.
On the press of a button, customers are added into the park and are allocated to different attractions in the park for a
pre-defined time period. In a way, we are simulating time-travel to generate the required data.

Our project is essentially the admin-dashboard of the amusement park.
A user with valid credentials, on logging in, is redirected to the page where the real-time statistics of the park are displayed in the form of bar-graphs, line-graphs and tables.
\thispagestyle{fancy}
